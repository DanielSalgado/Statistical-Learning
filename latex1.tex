\documentclass{article}\usepackage[]{graphicx}\usepackage[]{color}
%% maxwidth is the original width if it is less than linewidth
%% otherwise use linewidth (to make sure the graphics do not exceed the margin)
\makeatletter
\def\maxwidth{ %
  \ifdim\Gin@nat@width>\linewidth
    \linewidth
  \else
    \Gin@nat@width
  \fi
}
\makeatother

\definecolor{fgcolor}{rgb}{0.345, 0.345, 0.345}
\newcommand{\hlnum}[1]{\textcolor[rgb]{0.686,0.059,0.569}{#1}}%
\newcommand{\hlstr}[1]{\textcolor[rgb]{0.192,0.494,0.8}{#1}}%
\newcommand{\hlcom}[1]{\textcolor[rgb]{0.678,0.584,0.686}{\textit{#1}}}%
\newcommand{\hlopt}[1]{\textcolor[rgb]{0,0,0}{#1}}%
\newcommand{\hlstd}[1]{\textcolor[rgb]{0.345,0.345,0.345}{#1}}%
\newcommand{\hlkwa}[1]{\textcolor[rgb]{0.161,0.373,0.58}{\textbf{#1}}}%
\newcommand{\hlkwb}[1]{\textcolor[rgb]{0.69,0.353,0.396}{#1}}%
\newcommand{\hlkwc}[1]{\textcolor[rgb]{0.333,0.667,0.333}{#1}}%
\newcommand{\hlkwd}[1]{\textcolor[rgb]{0.737,0.353,0.396}{\textbf{#1}}}%

\usepackage{framed}
\makeatletter
\newenvironment{kframe}{%
 \def\at@end@of@kframe{}%
 \ifinner\ifhmode%
  \def\at@end@of@kframe{\end{minipage}}%
  \begin{minipage}{\columnwidth}%
 \fi\fi%
 \def\FrameCommand##1{\hskip\@totalleftmargin \hskip-\fboxsep
 \colorbox{shadecolor}{##1}\hskip-\fboxsep
     % There is no \\@totalrightmargin, so:
     \hskip-\linewidth \hskip-\@totalleftmargin \hskip\columnwidth}%
 \MakeFramed {\advance\hsize-\width
   \@totalleftmargin\z@ \linewidth\hsize
   \@setminipage}}%
 {\par\unskip\endMakeFramed%
 \at@end@of@kframe}
\makeatother

\definecolor{shadecolor}{rgb}{.97, .97, .97}
\definecolor{messagecolor}{rgb}{0, 0, 0}
\definecolor{warningcolor}{rgb}{1, 0, 1}
\definecolor{errorcolor}{rgb}{1, 0, 0}
\newenvironment{knitrout}{}{} % an empty environment to be redefined in TeX

\usepackage{alltt}
\IfFileExists{upquote.sty}{\usepackage{upquote}}{}
\begin{document}

Here is a code chunk

\begin{knitrout}
\definecolor{shadecolor}{rgb}{0.969, 0.969, 0.969}\color{fgcolor}\begin{kframe}
\begin{alltt}
\hlnum{1}\hlopt{+}\hlnum{1}
\end{alltt}
\begin{verbatim}
## [1] 2
\end{verbatim}
\begin{alltt}
\hlstd{lethers}
\end{alltt}


{\ttfamily\noindent\bfseries\color{errorcolor}{\#\# Error in eval(expr, envir, enclos): object 'lethers' not found}}\begin{alltt}
\hlkwd{par}\hlstd{(}\hlkwc{mar}\hlstd{=}\hlkwd{c}\hlstd{(}\hlnum{4}\hlstd{,}\hlnum{4}\hlstd{,}\hlnum{.2}\hlstd{,}\hlnum{.2}\hlstd{));} \hlkwd{plot}\hlstd{(}\hlkwd{rnorm}\hlstd{(}\hlnum{100}\hlstd{))}
\end{alltt}
\end{kframe}
\includegraphics[width=\maxwidth]{figure/foo-1} 

\end{knitrout}

Collect results from the template for each i and write them back later.




\section{Now j is 1}

\begin{knitrout}
\definecolor{shadecolor}{rgb}{0.969, 0.969, 0.969}\color{fgcolor}\begin{kframe}
\begin{alltt}
\hlkwd{print}\hlstd{(j)}
\end{alltt}
\begin{verbatim}
## [1] 1
\end{verbatim}
\begin{alltt}
\hlstd{iris[j, ]}
\end{alltt}
\begin{verbatim}
##   Sepal.Length Sepal.Width Petal.Length Petal.Width Species
## 1          5.1         3.5          1.4         0.2  setosa
\end{verbatim}
\end{kframe}
\end{knitrout}

\section{Now j is 2}

\begin{knitrout}
\definecolor{shadecolor}{rgb}{0.969, 0.969, 0.969}\color{fgcolor}\begin{kframe}
\begin{alltt}
\hlkwd{print}\hlstd{(j)}
\end{alltt}
\begin{verbatim}
## [1] 2
\end{verbatim}
\begin{alltt}
\hlstd{iris[j, ]}
\end{alltt}
\begin{verbatim}
##   Sepal.Length Sepal.Width Petal.Length Petal.Width Species
## 2          4.9           3          1.4         0.2  setosa
\end{verbatim}
\end{kframe}
\end{knitrout}

\section{Now j is 3}

\begin{knitrout}
\definecolor{shadecolor}{rgb}{0.969, 0.969, 0.969}\color{fgcolor}\begin{kframe}
\begin{alltt}
\hlkwd{print}\hlstd{(j)}
\end{alltt}
\begin{verbatim}
## [1] 3
\end{verbatim}
\begin{alltt}
\hlstd{iris[j, ]}
\end{alltt}
\begin{verbatim}
##   Sepal.Length Sepal.Width Petal.Length Petal.Width Species
## 3          4.7         3.2          1.3         0.2  setosa
\end{verbatim}
\end{kframe}
\end{knitrout}




We examine the relationship between speed and stopping
distance using a linear regression model:
$Y = \beta_0 + \beta_1 x + \epsilon$.

\begin{knitrout}
\definecolor{shadecolor}{rgb}{0.969, 0.969, 0.969}\color{fgcolor}\begin{kframe}
\begin{alltt}
\hlkwd{par}\hlstd{(}\hlkwc{mar} \hlstd{=} \hlkwd{c}\hlstd{(}\hlnum{4}\hlstd{,} \hlnum{4}\hlstd{,} \hlnum{1}\hlstd{,} \hlnum{1}\hlstd{),} \hlkwc{mgp} \hlstd{=} \hlkwd{c}\hlstd{(}\hlnum{2}\hlstd{,} \hlnum{1}\hlstd{,} \hlnum{0}\hlstd{),} \hlkwc{cex} \hlstd{=} \hlnum{0.8}\hlstd{)}
\hlkwd{plot}\hlstd{(cars,} \hlkwc{pch} \hlstd{=} \hlnum{20}\hlstd{,} \hlkwc{col} \hlstd{=} \hlstr{'darkgray'}\hlstd{)}
\hlstd{fit} \hlkwb{<-} \hlkwd{lm}\hlstd{(dist} \hlopt{~} \hlstd{speed,} \hlkwc{data} \hlstd{= cars)}
\hlkwd{abline}\hlstd{(fit,} \hlkwc{lwd} \hlstd{=} \hlnum{2}\hlstd{)}
\end{alltt}
\end{kframe}

{\centering \includegraphics[width=\maxwidth]{figure/model-1} 

}



\end{knitrout}

The slope of a simple linear regression is
3.9324088.

Probando como usar la funcion de child de knitr

\begin{knitrout}
\definecolor{shadecolor}{rgb}{0.969, 0.969, 0.969}\color{fgcolor}\begin{kframe}
\begin{alltt}
\hlkwd{options}\hlstd{(}\hlkwc{width} \hlstd{=} \hlnum{60}\hlstd{)}
\hlkwd{summary}\hlstd{(iris)}
\end{alltt}
\begin{verbatim}
##   Sepal.Length    Sepal.Width     Petal.Length  
##  Min.   :4.300   Min.   :2.000   Min.   :1.000  
##  1st Qu.:5.100   1st Qu.:2.800   1st Qu.:1.600  
##  Median :5.800   Median :3.000   Median :4.350  
##  Mean   :5.843   Mean   :3.057   Mean   :3.758  
##  3rd Qu.:6.400   3rd Qu.:3.300   3rd Qu.:5.100  
##  Max.   :7.900   Max.   :4.400   Max.   :6.900  
##   Petal.Width          Species  
##  Min.   :0.100   setosa    :50  
##  1st Qu.:0.300   versicolor:50  
##  Median :1.300   virginica :50  
##  Mean   :1.199                  
##  3rd Qu.:1.800                  
##  Max.   :2.500
\end{verbatim}
\end{kframe}
\end{knitrout}

Let's see how to work with child documents in knitr. Below we input
the file \textsf{knitr-input-child.Rnw}:


This chunk below is from the child document.

\begin{knitrout}
\definecolor{shadecolor}{rgb}{0.969, 0.969, 0.969}\color{fgcolor}\begin{kframe}
\begin{alltt}
\hlnum{1}\hlopt{+}\hlnum{1}
\end{alltt}
\begin{verbatim}
## [1] 2
\end{verbatim}
\begin{alltt}
\hlkwd{rnorm}\hlstd{(}\hlnum{5}\hlstd{)}
\end{alltt}
\begin{verbatim}
## [1] -0.33688488  0.75787814 -0.03341957 -2.78375575
## [5] -0.99935987
\end{verbatim}
\begin{alltt}
\hlkwd{plot}\hlstd{(}\hlnum{1}\hlstd{)}
\end{alltt}
\end{kframe}
\includegraphics[width=2in]{figure/test-child-1} 
\begin{kframe}\begin{alltt}
\hlkwd{boxplot}\hlstd{(}\hlnum{1}\hlopt{:}\hlnum{10}\hlstd{)}
\end{alltt}
\end{kframe}
\includegraphics[width=2in]{figure/test-child-2} 
\begin{kframe}\begin{alltt}
\hlkwd{str}\hlstd{(mtcars)}
\end{alltt}
\begin{verbatim}
## 'data.frame':	32 obs. of  11 variables:
##  $ mpg : num  21 21 22.8 21.4 18.7 18.1 14.3 24.4 22.8 19.2 ...
##  $ cyl : num  6 6 4 6 8 6 8 4 4 6 ...
##  $ disp: num  160 160 108 258 360 ...
##  $ hp  : num  110 110 93 110 175 105 245 62 95 123 ...
##  $ drat: num  3.9 3.9 3.85 3.08 3.15 2.76 3.21 3.69 3.92 3.92 ...
##  $ wt  : num  2.62 2.88 2.32 3.21 3.44 ...
##  $ qsec: num  16.5 17 18.6 19.4 17 ...
##  $ vs  : num  0 0 1 1 0 1 0 1 1 1 ...
##  $ am  : num  1 1 1 0 0 0 0 0 0 0 ...
##  $ gear: num  4 4 4 3 3 3 3 4 4 4 ...
##  $ carb: num  4 4 1 1 2 1 4 2 2 4 ...
\end{verbatim}
\end{kframe}
\end{knitrout}

Done!

\begin{knitrout}
\definecolor{shadecolor}{rgb}{0.969, 0.969, 0.969}\color{fgcolor}\begin{kframe}
\begin{alltt}
\hlkwd{read_chunk}\hlstd{(}\hlstr{'113-foo.R'}\hlstd{)}
\end{alltt}
\end{kframe}
\end{knitrout}


Probando con la externalizacion, ya se cargo \textsf{113-foo.R} en un chunk anterior, ahora con las etiquetas en el codigo empatando las etiquetas del siguiente chunk basta para insertar el codigo.
\begin{knitrout}
\definecolor{shadecolor}{rgb}{0.969, 0.969, 0.969}\color{fgcolor}\begin{kframe}
\begin{alltt}
\hlnum{1} \hlopt{+} \hlnum{1}
\end{alltt}
\begin{verbatim}
## [1] 2
\end{verbatim}
\begin{alltt}
\hlstd{x} \hlkwb{=} \hlkwd{rnorm}\hlstd{(}\hlnum{10}\hlstd{)}
\end{alltt}
\end{kframe}
\end{knitrout}

El siguiente chunk depende de las definiciones del primero.
\begin{knitrout}
\definecolor{shadecolor}{rgb}{0.969, 0.969, 0.969}\color{fgcolor}\begin{kframe}
\begin{alltt}
\hlkwd{mean}\hlstd{(x)}
\end{alltt}
\begin{verbatim}
## [1] 0.3009368
\end{verbatim}
\begin{alltt}
\hlkwd{sd}\hlstd{(x)}
\end{alltt}
\begin{verbatim}
## [1] 1.2003
\end{verbatim}
\end{kframe}
\end{knitrout}

Tablas

\begin{knitrout}
\definecolor{shadecolor}{rgb}{0.969, 0.969, 0.969}\color{fgcolor}\begin{kframe}
\begin{alltt}
\hlstd{n} \hlkwb{<-} \hlnum{100}
\hlstd{x} \hlkwb{<-} \hlkwd{rnorm}\hlstd{(n)}
\hlstd{y} \hlkwb{<-} \hlnum{2}\hlopt{*}\hlstd{x} \hlopt{+} \hlkwd{rnorm}\hlstd{(n)}
\hlstd{out} \hlkwb{<-} \hlkwd{lm}\hlstd{(y} \hlopt{~} \hlstd{x)}
\hlkwd{kable}\hlstd{(}\hlkwd{summary}\hlstd{(out)}\hlopt{$}\hlstd{coef,} \hlkwc{digits}\hlstd{=}\hlnum{2}\hlstd{)}
\end{alltt}
\end{kframe}
\begin{tabular}{l|r|r|r|r}
\hline
  & Estimate & Std. Error & t value & Pr(>|t|)\\
\hline
(Intercept) & 0.02 & 0.09 & 0.20 & 0.84\\
\hline
x & 2.04 & 0.10 & 21.01 & 0.00\\
\hline
\end{tabular}


\end{knitrout}

Con xtable
\begin{kframe}
\begin{alltt}
\hlstd{n} \hlkwb{<-} \hlnum{100}
\hlstd{x} \hlkwb{<-} \hlkwd{rnorm}\hlstd{(n)}
\hlstd{y} \hlkwb{<-} \hlnum{2}\hlopt{*}\hlstd{x} \hlopt{+} \hlkwd{rnorm}\hlstd{(n)}
\hlstd{out} \hlkwb{<-} \hlkwd{lm}\hlstd{(y} \hlopt{~} \hlstd{x)}
\hlkwd{library}\hlstd{(xtable)}
\hlkwd{xtable}\hlstd{(}\hlkwd{summary}\hlstd{(out)}\hlopt{$}\hlstd{coef,} \hlkwc{digits}\hlstd{=}\hlkwd{c}\hlstd{(}\hlnum{0}\hlstd{,} \hlnum{2}\hlstd{,} \hlnum{2}\hlstd{,} \hlnum{1}\hlstd{,} \hlnum{2}\hlstd{))}
\end{alltt}
\end{kframe}% latex table generated in R 3.1.2 by xtable 1.7-4 package
% Sun Feb 15 13:31:54 2015
\begin{table}[ht]
\centering
\begin{tabular}{rrrrr}
  \hline
 & Estimate & Std. Error & t value & Pr($>$$|$t$|$) \\ 
  \hline
(Intercept) & 0.13 & 0.10 & 1.3 & 0.19 \\ 
  x & 1.95 & 0.09 & 22.3 & 0.00 \\ 
   \hline
\end{tabular}
\end{table}

\end{document}
